%-------------------------------------------------------------------------------
%-------------------------------------------------------------------------------
%-------------------------------------------------------------------------------
%-------------------------------------------------------------------------------
% Дополнительные комманды для данной работы:

\newcommand{\rb}[1]{ \left( #1 \right) }

\newcommand{\sqbr}[1]{ \left[ #1 \right]}

\newcommand{\InsertImage}[2]
{
    \begin{center}
        \includegraphics[width=#1\textwidth]{#2}        
    \end{center}
}

\newcommand{\InsertTwoImages}[3]
{
    \begin{center}
        \includegraphics[width=#1\textwidth]{#2}
        \includegraphics[width=#1\textwidth]{#3}
    \end{center}
}

\newcommand{\InsertFourImages}[5]
{
    \begin{center}
        \includegraphics[width=#1\textwidth]{#2}
        \includegraphics[width=#1\textwidth]{#3}
    \end{center}
    \begin{center}
        \includegraphics[width=#1\textwidth]{#4}
        \includegraphics[width=#1\textwidth]{#5}
    \end{center}
}


%-------------------------------------------------------------------------------
%-------------------------------------------------------------------------------
%-------------------------------------------------------------------------------
\insertTitle{Планирование и анализ эксперимента}
{«Синтез  непрерывных A-оптимальных  планов  эксперимента

для  нечетких квадратичных однофакторных моделей

с двумя подобластями определения»}
{ПМ-63}{Кожекин М.В.}{5}{2020}


%-------------------------------------------------------------------------------
\section{Задание}

1. Ознакомиться с математическим аппаратом построения регрессионных моделей в рамках
концепции нечетких систем, вопросами оптимального планирования эксперимента.

2. Разработать программное приложение синтеза  непрерывных A-оптимальных планов
эксперимента для однофакторных моделей.

3. Работа приложения должна быть продемонстрирована на нескольких тестовых примерах.
Оптимальность полученных планов должна быть подтверждена выполнением соответствующих
условий с приемлемой точностью.



%-------------------------------------------------------------------------------
\section{Требования}

Приложение  должно  осуществлять  синтез непрерывных A-оптимальных планов 
эксперимента для нечетких регрессионных моделей с одним вещественным фактором.

Область  определения  вещественной  переменной (интервал  [–1; +1]) при фаззификации 
разбивается на две трапецевидные нечеткие партиции с функциями принадлежности:


\[
    \mu_1(x) = 
        \begin{cases}
            1,                          & \quad x \leq -\Delta \\
            \frac{\Delta-x}{2\Delta},   & \quad -\Delta \leq x \leq \Delta \\
            0,                          & \quad x \geq \Delta \\
        \end{cases}
\]
\[ \mu_2(x) = 1 - \mu_1(x) \]

Построить оптимальные планы для значений $\Delta = {0.5;0.4;0.3;0.2}$.
Базовая модель квадратичная. Характеристики построенных планов представить в таблице,
координаты точек спектра планов отобразить на рисунке вместе с функциями принадлежности.




%-------------------------------------------------------------------------------
\section{Анализ}
\subsection{Нечёткая логика}

В исходном виде квадратичная однофакторная модель по своему списку регрессоров имеет вид:
\[ f^T(x) = (1, x_1, x_1^2, \mu_1(x_1), \mu_2(x_1), \mu_1(x_1)x_1, \mu_2(x_1)x_1, \mu_1(x_1)x_1^2, \mu_2(x_1)x_1^2) \]

Редуцируем регрессоры, связанные со \textbf{второй} партицией. Тогда модель имеет вид:
\[ f^T(x) = (1, x_1, x_1^2, \mu_1(x_1), \mu_1(x_1)x_1, \mu_1(x_1)x_1^2 \]
\InsertImage{0.55}{../pics/mu.png}




\subsection{Последовательный алгоритм}

1. Выбирается невырожденный начальный план $\varepsilon^0$. Номер итерации s = 0.

2. Отыскивается точка глобального эстремума $x^s$:
\[ x^s = arg \min_{x \in \tilde{X}}  \varphi(x, \varepsilon^s), \text{ где } \quad \varphi(x, \varepsilon) = f^T(x) \frac{d\Phi [M(\varepsilon)] }{ dM(\varepsilon) } f(x) \]

3. Проверяется приближенное выполнение необходимых и достаточных условий оптимальности планов
\[ \left| - \min_{x \in \tilde{X}}  \varphi(x, \varepsilon^s)
 + tr M(\varepsilon^s)  \frac{d\Phi [M(\varepsilon)] }{ dM(\varepsilon) } \right| \leq \delta,
\text{ где} \quad \delta = \left| \min_{x \in \tilde{X}}  \varphi(x, \varepsilon^s) \right| \times 0.01 \]
Если условие выполнено, то работа алгоритма прекращается. В противном случае осуществляется переход на шаг 4.

4. Составляется план
\[ \varepsilon^{s+1} = (1 - \alpha^s) \varepsilon^s + \alpha^s \varepsilon(x^s)\]
где $\alpha \in (0,1)$, $\varepsilon(x^s)$ - план, состоящий из одной точки $x^s$.

5. Величина $\Psi[M(\varepsilon^{s+1})]$ сравнивается с величиной $\Psi[M(\varepsilon^{s})]$:

а) если $\Psi[M(\varepsilon^{s+1})] \geq \Psi[M(\varepsilon^{s})]$, то $\alpha^s$ уменьшается в
$\gamma$ раз и повторяются шаги 4-5;

б) если имеет место обратное неравенство, то s = s+1 и происходит переход на шаг 2.
\vspace{5mm}



В конце происходит процедура "очистки" плана.

1. Точки, тяготеющие к одной из групп, объединяются по правилу
\[ p_j^s = \sum_{k=1}^l p_{j_k}^s, \quad x_j^s = \frac{\sum_{k=1}^l {x_{j_k}^s p_{j_k}^s}}{p_j^s} \]

2. Точки с малыми весами, не тяготеющие ни к одной из групп, указанных в 4), выбрасываются.
Их веса распределяются между остальными точками 


\subsection{Критерий A-оптимальности и проиводная}
\[ \varepsilon^* = Arg \min_\varepsilon tr \rb{M^{-1} \rb{\varepsilon} } \]
\[ \frac{\partial \Psi \sqbr{ M\rb{\varepsilon^s} } }
{\partial M\rb{\varepsilon^s}} = - M^{-2} \rb{\varepsilon^s} \]



%-------------------------------------------------------------------------------
%-------------------------------------------------------------------------------
\section{Работа алгоритма}

\subsection{Параметры начального приближения и алгоритма}

Точки спектра плана находятся в интервале [-1, +1] с шагом 0.004. 
Всего у нас их 501 штука, вес каждой равен 1/501. 
Максимальное число итераций по s = 200, максимальное число итераций по $\alpha$ = 30.

\subsection{Асимметрия функции fi}

Для начала рассмотрим значения функции $\varphi \rb{x, \varepsilon^s}$ на начальной итерации.
Как мы видим, для каждого значения $\Delta$ функция асимметрична. Это может быть вызвано 
ошибками численного вычисления производной по функционалу. Символьное вычисление могло бы исправить эту ситуацию.

\InsertFourImages{0.49}
{../pics/fi_delta_0.2.png}
{../pics/fi_delta_0.3.png}
{../pics/fi_delta_0.4.png}
{../pics/fi_delta_0.5.png}


\subsection{Очистка плана}

В данном алгоритме мы производим очистку плана в 3 этапа:

1. Объединение повторяющихся точек.

2. Удаление малозначительных ($p_i$ < 0.00009) точек и пересчёт весов.

3. Объединение точек в радиусе 0.1 используя центр масс

\vspace{10mm}


Было проведено 8 эксперименов: 4 с очисткой плана и 4 без.
Очистка проиводилась каждые 10 шагов алгоритма, начиная с 100-й итерации.

\InsertFourImages{0.49}
{../pics/research_clear_0.2.png}
{../pics/research_clear_0.3.png}
{../pics/research_clear_0.4.png}
{../pics/research_clear_0.5.png}

\textbf{Вывод:}

Алгоритм без очистки плана сходится плавно, но медленно.
Очистка спектра плана позволяет найти на 15-30\% более А-оптимальный план, 
имеющий в 35-70 раз меньше точек. Также это позволяет ускорить вычисления.


%-------------------------------------------------------------------------------
%-------------------------------------------------------------------------------
\section{Исследование влияния параметра $\Delta$}

\InsertImage{0.6}
{../pics/convergence_delta_yes_clear.png}

Как видно из графиков наиболее оптимальным значением параметра $\Delta$ является 0.2.

При нём функционал $\Psi \approx 2200$ хотя при $Delta = 0.4$ он в 2.5 раза больше $\Psi \approx 5900$.

Получившиеся в результате работы алгоритма планы:


\InsertFourImages{0.48}
{../pics/plan_delta_clear_0.2_s_190.png}
{../pics/plan_delta_clear_0.3_s_170.png}
{../pics/plan_delta_clear_0.4_s_189.png}
{../pics/plan_delta_clear_0.5_s_110.png}







%-------------------------------------------------------------------------------
\section{Исходный код программы}
\myCodeInput{python}{main.py}{../main.py}