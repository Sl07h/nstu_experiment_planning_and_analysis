%-------------------------------------------------------------------------------
%-------------------------------------------------------------------------------
%-------------------------------------------------------------------------------
% Дополнительные комманды для данной работы:

\newcommand{\rb}[1]{ \left( #1 \right) }

\newcommand{\sqbr}[1]{ \left[ #1 \right]}

\newcommand{\InsertImage}[2]
{
    \begin{center}
        \includegraphics[width=#1\textwidth]{#2}        
    \end{center}
}

\newcommand{\InsertFourImages}[5]
{
    \begin{center}
        \includegraphics[width=#1\textwidth]{#2}
        \includegraphics[width=#1\textwidth]{#3}
    \end{center}
    \begin{center}
        \includegraphics[width=#1\textwidth]{#4}
        \includegraphics[width=#1\textwidth]{#5}
    \end{center}
}


%-------------------------------------------------------------------------------
%-------------------------------------------------------------------------------
%-------------------------------------------------------------------------------
\insertTitle{Планирование и анализ эксперимента}
{«Синтез  дискретных A-оптимальных  планов эксперимента 

для нечетких линейных двухфакторных моделей 

с сигмоидными функциями принадлежности»}
{ПМ-63}{Утюганов Д.С.}{22}{2020}



%-------------------------------------------------------------------------------
%-------------------------------------------------------------------------------
\section{Задание}

1. Ознакомиться с математическим аппаратом построения регрессионных моделей в рамках
концепции нечетких систем, вопросами оптимального планирования эксперимента.

2. Разработать  программное  приложение  синтеза дискретных A-оптимальных планов 
эксперимента  для  двухфакторных  моделей. В  качестве  алгоритма использовать
градиентный алгоритм \textbf{замены точек}.

3. Работа приложения должна быть продемонстрирована на нескольких тестовых примерах.



%-------------------------------------------------------------------------------
%-------------------------------------------------------------------------------
\section{Требования}

Приложение должно осуществлять синтез дискретных A-оптимальных планов эксперимента
для нечетких регрессионных  моделей  с  двумя  вещественными факторами.

Область  определения  каждой  вещественной  переменной (интервал  [–1;+1])
при фаззификации разбивается на две сигмоидные нечеткие партиции с функциями принадлежности:
\[ \mu_2(x) = \frac{1}{1+e^{-d_2(x-d_1)}} \]
\[ \mu_1(x) = 1 - \mu_2(x) \]



Построить оптимальные планы для  значений $d_1 = 0; d_2 = \{8; 12; 16; 20\}$.
Базовая  модель  линейная.  Область планирования - регулярная  сетка  21х21.
Число  точек  в  плане  20,  30,  40. Характеристики  построенных планов представить
в таблице, координаты  точек спектра планов отобразить на рисунке и в виде таблицы.



%-------------------------------------------------------------------------------
%-------------------------------------------------------------------------------
\section{Анализ}

\subsection{Нечёткая логика}

В исходном виде линейная двухфакторная модель по своему списку регрессоров имеет вид:
\[ f^T(x) = (1, x_1, x_2, \mu_1(x_1), \mu_2(x_1), \mu_1(x_2), \mu_2(x_2), \]
\[ \mu_1(x_1)x_1, \mu_2(x_1)x_1, \mu_1(x_2)x_1, \mu_2(x_2)x_1, \]
\[ \mu_1(x_1)x_2, \mu_2(x_1)x_2, \mu_1(x_2)x_2, \mu_2(x_2)x_2) \]

Редуцируем регрессоры, связанные со \textbf{второй} партицией. Тогда модель имеет вид:
\[f^T(x) = \rb{1, x_1, x_2, \mu_1(x_1), \mu_1(x_2),
\mu_1(x_1)x_1, \mu_1(x_2)x_1, 
\mu_1(x_1)x_2, \mu_1(x_2)x_2 }\]
\InsertImage{0.55}{../pics/mu.png}



\subsection{Критерий A-оптимальности и проиводная}
\[ \varepsilon^* = Arg \min_\varepsilon tr \rb{M^{-1} \rb{\varepsilon} } \]
\[ \frac{\partial \Psi \sqbr{ M \rb \varepsilon^s_N } }
{\partial M\rb{\varepsilon_N}} = \rb{M^{-2} \rb{\varepsilon}}^T \]



\subsection{Градиентный алгоритм}

1. Выбирается невырожденный начальный план $\varepsilon^0_N$, s = 0.

2. Вычисляются элементы вектора градиента
\[ \phi \rb{x_j, \varepsilon^s_N} = f^T(x_j) 
\frac{\partial \Psi \sqbr{ M \rb \varepsilon^s_N } }
{\partial M\rb{\varepsilon_N}} f(x_j), \quad j=1 \ldots n \]
для плана $\varepsilon^s_N$, $\varepsilon^s_{N, i} = \varepsilon^s_N$.
счётчик числа проведённых замен точек при движении по вычисленному направлению
градиента устанавливается в 0 (i=1).


3. Выбирается точка $x^*$ на множестве $\tilde{X}$ по правилу
\[x^* = Arg \max_{x \in \tilde{X}} \phi \rb{x, \varepsilon^s_N} \]

4. Среди точек плана $\varepsilon^s_{N,i}$ выбирается точка $x^{**}$ по правилу
\[x^{**} = Arg \min_{x_j \in \varepsilon^s_{N, i}} \phi \rb{x_j, \varepsilon^s_N} \]

5. Точка $x^{**}$ заменяется в плане $\varepsilon^s_{N,i}$ на точку $x^*$.
В результате формируется план $\varepsilon^s_{N,i+1}$.

6. Сравниваются величины $\Psi \sqbr{ M \rb{ \varepsilon^s_{N,i+1}} }$ и  $\Psi \sqbr{ M \rb{ \varepsilon^s_{N,i}} }$

а) если  $\Psi \sqbr{ M \rb{ \varepsilon^s_{N,i+1}} } > \Psi \sqbr{ M \rb{ \varepsilon^s_{N,i}} }$,
то счётчик i проведённых удачных замен точек увеличивается на единицу и осуществляется переход дна шаг 3,
при этом точки $x^{**}$ и $x^{*}$ исключаются из рассмотрения;

б) в противном случае: если i = 0, то вычисления прекращаются, иначе -- s заменяется на s+1
и осуществляется переход на шаг 2.




%-------------------------------------------------------------------------------
%-------------------------------------------------------------------------------
\section{Исследование влияния числа точек плана N}

\subsection{N = 20}

\InsertImage{0.58}{../pics/research_N_20.png}

\InsertImage{0.58}{../pics/convergence_grad_alg_N_20_d2_20.png}

\InsertImage{0.58}{../pics/plan_grad_alg_N_20_s_40_d2_20.png}



\subsection{N = 30}

\InsertImage{0.58}{../pics/research_N_30.png}

\InsertImage{0.58}{../pics/convergence_grad_alg_N_30_d2_20.png}

\InsertImage{0.58}{../pics/plan_grad_alg_N_30_s_40_d2_20.png}



\subsection{N = 40}

\InsertImage{0.58}{../pics/research_N_40.png}

\InsertImage{0.58}{../pics/convergence_grad_alg_N_40_d2_20.png}

\InsertImage{0.58}{../pics/plan_grad_alg_N_40_s_40_d2_20.png}



\subsection{Вывод}

С ростом числа точек плана от 20 до 40  план становится более A-оптимальным.
На 30 итерации $tr(M^{-1})$ значение падает со 160 до 140.

Можно заметить, что остались точки, с координатами близкими к -1, 0, 1.




%-------------------------------------------------------------------------------
%-------------------------------------------------------------------------------
\section{Исседование параметра $d_2$}

\InsertFourImages{0.49}
{../pics/plan_grad_alg_N_30_s_40_d2_8.png}
{../pics/plan_grad_alg_N_30_s_40_d2_12.png}
{../pics/plan_grad_alg_N_30_s_40_d2_16.png}
{../pics/plan_grad_alg_N_30_s_40_d2_20.png}

\subsection{Вывод:}

Как видно, при росте параметра $d_2$ точки стремятся к осям и началу координат.

Критерий A-оптимальности плана уменьшилась почти в 2 раза с 239.87 до 140.74.

\newpage



%-------------------------------------------------------------------------------
%-------------------------------------------------------------------------------
\section{Исходный код программы}
\myCodeInput{python}{main.py}{../main.py}